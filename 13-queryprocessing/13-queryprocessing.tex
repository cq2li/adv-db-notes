\documentclass[11pt]{article}

\newcommand{\lecturenumber}{13}
\newcommand{\lecturename}{Query Execution \& Processing}
\newcommand{\lecturedata}{2018-03-05}

\usepackage{../dbnotes}

\begin{document}

\maketitle
\thispagestyle{plain}

%% ==================================================================
%% Operator Execution
%% ==================================================================
\section{Operator Execution}
Since an in-memory DMBS has all the data already in DRAM, the disk I/O is no longer the main 
bottleneck. Instead, other issues including the query execution become the main problems. When
executing a query, the DBMS converts a SQL statement into a query plan, where physical or logical 
operators are arranged in a tree. Operator execution focuses on the problem of efficiently 
executing these operators of a query plan with their corresponding data. 


There are three techniques of speeding up the queries:
\begin{itemize}
    \item \textbf{Reduce Instruction Count}: 
    Use fewer instructions to do the same amount of work. 
    Usually by specializing or organizing the code to reduce the number of instructions to execute 
    the operators. 
    \item \textbf{Reduce Cycles per Instruction}: 
    Execute more CPU instructions in fewer cycles. 
    This means to maximize locality by reducing cache misses and stalls due to memory load/stores.
    \item \textbf{Parallelize Execution}: 
    Use multiple threads to compute each query in parallel.
\end{itemize}

%% ==================================================================
%% MonetDB/X100
%% ==================================================================
\section{MonetDB/X100 Analysis}
The paper by Boncz et al~\cite{Boncz2005} provides a low-level analysis of execution  
bottlenecks for in-memory DBMSs on OLAP workloads. Their results show how DBMSs at that time were 
designed incorrectly for modern CPU architectures.

%% ----------------------------------------------------
%% CPU Overview
%% ----------------------------------------------------
\subsection*{CPU Overview for Databases}
CPUs organize instructions into \textbf{pipeline stages}, whose goal is to keep all parts of the 
processor busy at each cycle by masking delays from instructions that cannot complete in a single 
cycle. Super-scalar CPUs support multiple pipelines. They can execute multiple instructions in 
parallel in a single cycle if the instructions are independent. It is categorized as ``Single 
Instruction stream, Single Data stream (\textbf{SISD})'' in Flynn's Taxonomy.

%% ----------------------------------------------------
%% DBMS/CPU Problems
%% ----------------------------------------------------
\subsection*{DBMS/CPU Problems}
For DBMSs, there are two problems with CPU pipelining. 

\begin{itemize}
	\item \textbf{Dependencies}: 
	If one instruction depends on another instruction, 
	then the CPU cannot push it immediately into the same pipeline. 
	
	\item \textbf{Branch Prediction}: 
	The CPU tries to predict what branch the program will take and fill in the pipeline with its 
	instructions. If it gets it wrong, then it has to throw away any speculative work and flush the 
	pipeline. Two examples are \textbf{Branch Misprediction} caused by selection scan and 
	\textbf{Excessive Instructions} caused by the support of different data types.
\end{itemize}

%% ==================================================================
%% Processing Model 
%% ==================================================================
\section{Processing Model}
A DBMS's processing model defines how the system executes a query plan. There are different 
trade-offs for different workloads. In addition, there are two plan processing directions:
\begin{itemize}
	\item \textbf{Top-to-Bottom}: 
	Start with the root and ``pull'' data up from its children. 
	This approach always passe tuples with function calls.
	\item \textbf{Bottom-to-Top}: 
	Start with leaf nodes and push data to their parents. 
	This approach allows for tighter control of caches/registers in pipelines.
\end{itemize} 


%% ----------------------------------------------------
%% Iterator Model
%% ----------------------------------------------------
\subsection*{Iterator Model}
The iterator model is also called the \textbf{Volcano} or \textbf{Pipeline} Model. Each query plan 
operator implements a next function. On each invocation of the next function, the operator returns 
either a single tuple or a null marker if there are no more tuples. The operator implements a loop 
that calls next on its children to retrieve their tuples and then process them.

This model is used in almost every DBMS as a general approach. It allows for tuple pipelining, but 
some operators have to block until their children emit all of their tuples, such as ``joins'', 
``subqueries'', and ``order by''. Output control works easily with this approach.

%% ----------------------------------------------------
%% Materialization Model
%% ----------------------------------------------------
\subsection*{Materialization Model}
In the materialization model, each operator processes its input all at once and then emits its 
output all at once. The operator ``materializes'' its output as a single result. The DBMS can push 
down hints to avoid scanning too many tuples and return only a single column. The output can be 
either whole tuples (NSM) or subsets of columns (DSM). 

Since this approach has lower execution / coordination overhead and fewer function calls, it is 
good for OLTP workloads as the  queries only access a small number of tuples at a time. On the 
other hand, it is not good for OLAP queries with large intermediate results.

%% ----------------------------------------------------
%% Vectorized / Batch Model
%% ----------------------------------------------------
\subsection*{Vectorized / Batch Model}
In the vectorized model, each operator implements a next function like the iterator model but emits 
a batch of tuples instead of a single one. The operator's internal loop processes multiple tuples 
at a time. The size of the batch can vary based on hardware or query properties.

The vectorized model is considered ideal for OLAP queries because it greatly reduces the number of 
invocations per operator. It allows operators to use vectorized (SIMD) instructions to process 
batches of tuples.


%% ==================================================================
%% Parallel Execution
%% ==================================================================
\section{Parallel Execution}
There are two types of parallelisms:
%% ----------------------------------------------------
%% Inter-Query Parallelism
%% ----------------------------------------------------
\subsection*{Inter-Query Parallelism}
Inter-Query Parallelism improves the overall performance of query execution by allowing multiple 
queries to execute simultaneously. It provides the illusion of isolation through concurrency 
control scheme. The difficulty of implementing a concurrency control scheme is not significantly 
affected by the DBMS’s process model.

%% ----------------------------------------------------
%% Intra-Query Parallelism
%% ----------------------------------------------------
\subsection*{Intra-Query Parallelism}
Intra-Query Parallelism improves the performance of a single query by executing its operators in 
parallel. There are two approaches that are not mutually exclusive. 
\begin{itemize}
    \item \textbf{Intra-Operator (Horizontal)}:
    This approach decomposes operators into independent instances that perform the same function on 
    different subsets of data. The DBMS inserts an exchange operator into the query plan to 
    coalesce results from children operators.
    \item \textbf{Inter-Operator (Vertical)}:
    This technique overlaps operations in order to pipeline data from one stage to the next without 
    materialization. The technique is also called \textbf{pipelined parallelism}. This approach is 
    not widely used in traditional relational DBMSs since not all operators can emit output until 
    they have seen all of the tuples from their children. It is more common 
    in \textbf{stream processing} systems such as kafka and Apache Spark.
\end{itemize}

%% ==================================================================
%% Worker Allocation
%% ==================================================================
\section*{Worker Allocation}
Determining the right number of workers to use for a query plan depends on the number of CPU 
cores, the size of the data, and the functionality of the operators. 
We may assign \textbf{One Worker per Core}, where each core is assigned one thread that is pinned 
to that core in the OS, or \textbf{Multiple Workers per Core}, which uses a pool of workers per 
core (or per socket) to fully utilize CPU cores in case one worker at a core blocks.

%% ==================================================================
%% Task Assignment
%% ==================================================================
\section*{Task Assignment}
In query execution, a \textbf{task} is the execution of a sequence of one or more operator 
instances. A \textbf{worker} is the DBMS component that is responsible for executing tasks on 
behalf of the client and returning the results.
There are two ways to assign tasks to workers:
\begin{itemize}
    \item \textbf{Push}: 
    This method uses a centralized dispatcher to assign tasks to workers and monitors their progress. 
    When the worker notifies the dispatcher that it is finished, it is given a new task.
    \item \textbf{Pull}: 
    In this approach, workers pull the next task from a queue, process it, and then return to get the 
    next task.
\end{itemize}

% ==================================================================
% BIBLIOGRAPHY
% ==================================================================
\newpage
\bibliographystyle{abbrvnat}
\bibliography{13-queryprocessing}

\end{document}

