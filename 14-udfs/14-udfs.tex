\documentclass[11pt]{article}

\newcommand{\lectureNum}{24}
\newcommand{\lectureName}{Server-side Logic Execution}
\newcommand{\lectureDate}{2020-04-22}

\usepackage{../dbnotes}

\begin{document}

\maketitle
\thispagestyle{plain}

%% ==================================================================
%% Embedded Database Logic
%% ==================================================================
\section{Embedded Database Logic}
Move application logic into the DBMS to avoid multiple network round-trips and to extend the functionality of the DBMS. Potential Benefits: Efficiency \& Reuse.
There are several different types of Embedded Database Logic inculding User-Defined Functions (UDFs), Stored Procedures, Triggers, User-Defined Types (UDTs), and User-Defined Aggregates (UDAs).

%% ==================================================================
%% User-Defined Functions
%% ==================================================================
\section{User-Defined Functions}
A user-defined function (UDF) is a function written by the application developer that extends the system's functionality beyond its built-in operations. It takes in input arguments (scalars), performs some computation, and then returns a result (scalars, tables)

%% ----------------------------------------------------
%% UDF Advantages
%% ----------------------------------------------------
\subsection*{UDF Advantages}
\begin{itemize}
	\item UDFs encourage modularity and code reuse: different queries can reuse the same application logic without having to reimplement it each time.
	
	\item Fewer network round-trips between application server and DBMS for complex operations.
	
	\item Some types of application logic are easier to express and read as UDFs than SQL.
\end{itemize}

%% ----------------------------------------------------
%% UDF Disadvantages
%% ----------------------------------------------------
\subsection*{UDF Disadvantages}
\begin{itemize}
	\item Query optimizers treat UDFs as black boxes. 
	Unable to estimate cost if you don't know what a UDF is going to do when you run it.
	
	\item It is difficult to parallelize UDFs due to correlated queries inside of them.
	Some DBMSs will only execute queries with a single thread if they contain a UDF.
	Some UDFs incrementally construct queries.
	
	\item Complex UDFs in SELECT / WHERE clauses force the DBMS to execute iteratively.
	RBAR = "Row By Agonizing Row"
	Things get even worse if UDF invokes queries due to implicit joins that the optimizer cannot "see".
	
	\item Since the DBMS executes the commands in the UDF one-by-one, it is unable to perform crossstatement optimizations.
\end{itemize}

%% ==================================================================
%% Froid
%% ==================================================================
\section{Froid}
Automatically convert UDFs into relational expressions that are inlined as sub-queries~\cite{Ramachandra2017}. Does not require the app developer to change UDF code. Perform conversion during the rewrite phase to avoid having to change the cost-base optimizer. Commercial DBMSs already have powerful transformation rules for executing sub-queries efficiently.
Froid has five main steps:
\begin{itemize}
	\item Transform Statements
	\item Break UDF into Regions
	\item Merge Expressions
	\item Inline UDF Expression into Query
	\item Run Through Query Optimizer
\end{itemize}

%% ----------------------------------------------------
%% Sub-Queries
%% ----------------------------------------------------
\subsection*{Sub-Queries}
The DBMS treats nested sub-queries in the where clause as functions that take parameters and return a single value or set of values. The two approaches are:
\begin{itemize}
	\item Rewrite to de-correlate and/or flatten them
	\item Decompose nested query and store result to temporary table. Then the outer joins with the temporary table.
\end{itemize}

%% ----------------------------------------------------
%% Lateral Join
%% ----------------------------------------------------
\subsection*{Lateral Join}
A lateral inner subquery can refer to fields in rows of the table reference to determine which rows to return. Allows you to have sub-queries in FROM clause. The DBMS iterates through each row in the table referenced and evaluates the inner sub-query for each row. The rows returned by the inner sub-query are added to the result of the join with the outer query. 

%% ==================================================================
%% UDF CTE Conversion
%% ==================================================================
\section{UDF CTE Conversion}
Rewrite UDFs into plain SQL commands~\cite{Duta2019}. Use recursive common table expressions (CTEs) to support iterations and other control flow concepts not supported in Froid. DBMS Agnostic can be implemented as a rewrite middleware layer on top of any DBMS that supports CTEs. The five main steps to conver UDFs to CTEs are:
\begin{itemize}
	\item Static Single Assignment Form
	\item Administrative Normal Form
	\item Mutual to Direct Recursion
	\item Tail Recursion to WITH RECURSIVE
	\item Run Through Query Optimizer
\end{itemize}

% ==================================================================
% BIBLIOGRAPHY
% ==================================================================
\newpage
\bibliographystyle{abbrvnat}
\bibliography{14-udfs}

\end{document}
