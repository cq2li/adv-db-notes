\documentclass[11pt]{article}

\newcommand{\lecturenumber}{16}
\newcommand{\lecturename}{Vectorization vs. Compilation}
\newcommand{\lecturedata}{2018-01-24}

\usepackage{../dbnotes}

\begin{document}

\maketitle
\thispagestyle{plain}

%% ==================================================================
%% Observation
%% ==================================================================
\section{Observation}
Vectorization can speed up query performance.Compilation can speed up query performance. We have 
not discussed which approach is better and under what conditions.
\begin{itemize}
    \item 
    \dbSys{Vectorwise} - Precompiled Primitives
    
    Pre-compiles thousands of “primitives” that perform basic operations on typed data~\cite{Rundefinedducanu2013}. Using simple kernels for each primitive means that they are easier to vectorize.
    
    The DBMS then executes a query plan that invokes these primitives at runtime.→Function calls 
    are amortized over multiple tuples
    
    \item
    \dbSys{HyPer} - JIT Query Compilation
    
    Compile queries in-memory into native code using the LLVM toolkit~\cite{Neumann2011}.
    
    Organizes query processing in a way to keep a tuple in CPU registers for as long as possible.  
    Bottom-to-top / push-based query processing model. Not vectorizable (as originally described).
\end{itemize}

%% ==================================================================
%% Vectorization vs. Compilation
%% ==================================================================
\section{Vectorization vs. Compilation}
Single test-bed system to analyze the trade-offs between vectorized execution and query 
compilation~\cite{Kersten2018}. Implemented high-level algorithms the same in each system but 
varied the implementation details. Example: Murmur2 vs. CRC Hash Functions
\begin{itemize}
    \item 
    \textbf{Approach \#1: Tectorwise}:
    Break operations into pre-compiled primitives.
    Have to materialize the output of primitives at each step. 
    
    \item 
    \textbf{Approach \#2: Typer}:
    Push-based processing model with JIT compilation.
    Process a single tuple up entire pipeline without materializing the intermediate results.
\end{itemize}
    
%% ----------------------------------------------------
%% Main Findings
%% ----------------------------------------------------
\subsection*{Main Findings}
Both models are efficient and achieve roughly the same performance. Data-centric is better for 
computational queries with few cache misses. Vectorization is slightly better at hiding cache miss 
latencies.
\begin{itemize}
	\item 
	\textbf{SIMD Performance}
	
	Evaluate vectorized branchless selection and hash probe in Tectorwise. They use AVX-512 because 
	it includes new instructions to make it easier to implement algorithms using vertical 
	vectorization. 
	
	\item 
	\textbf{Auto-Vectorization}
	
	Measure how well the compiler is able to vectorize the Vectorwiseprimitives. 
	Targets: GCC v7.2, Clang v5.0, ICC v18.
	
	ICC was able to vectorize the most primitives using AVX-512:
	Vectorized: Hashing, Selection, Projection
	Not Vectorized: Hash Table Probing, Aggregation
\end{itemize}

The paper (partially) assumes that vectorization and compilation are mutually exclusive. \dbSys{HyPer} fuses operators together so that they work on a single tuple a time to maximize CPU register reuse and minimize cache misses.

%% ==================================================================
%% Relaxed Operator Fusion
%% ==================================================================
\section{Relaxed Operator Fusion}
Each pipeline fuses operators together into loop. Each pipeline is a tuple-at-a-time process.
Fusion inhibits some optimizations:→Unable to look ahead in tuple stream. Unable to overlap 
computation and memory access.

%% ----------------------------------------------------
%% Relaxed Operator Fusion
%% ----------------------------------------------------
\subsection*{Relaxed Operator Fusion}
Vectorized processing model designed for query compilation execution engines~\cite{Menon2017}. Decompose pipelines into stages that operate on vectors of tuples. Each stage may contain multiple operators. Communicate through cache-resident buffers. Stages are granularity of vectorization + fusion.

%% ----------------------------------------------------
%% ROF Prefetching
%% ----------------------------------------------------
\subsection*{ROF Prefetching}
The DBMS can tell the CPU to grab the next vector while it works on the current batch. Prefetch-enabled operators define start of new stage. Hides the cache miss latency.
Any prefetching technique is suitable→Group prefetching, software pipelining, AMAC. Group prefetching works and is simple to implement.

%% ==================================================================
%% Conclusion
%% ==================================================================
\section{Conclusion}
No major performance difference between the \dbSys{Vectorwise} and \dbSys{HyPer} approaches for all queries. 
ROF combines vectorization and compilation into a hybrid query processing model. Trades off additional instructions for reduced CPI.
% ==================================================================
% BIBLIOGRAPHY
% ==================================================================
\newpage
\bibliographystyle{abbrvnat}
\bibliography{16-vectorization2}



\end{document}
