\documentclass[11pt]{article}

\newcommand{\lecturenumber}{8}
\newcommand{\lecturename}{OLTP Indexes - Part 1}
\newcommand{\lecturedata}{2018-02-12}

\usepackage{../dbnotes}

\begin{document}

\maketitle
\thispagestyle{plain}

\section{T-Trees ~\cite{P294}}
\begin{itemize}
    \item Based on AVL Trees. Instead of storing keys in nodes, store pointers to their original values
    \item Proposed in 1986 from Univ. of Wisconsin
    \item Used in TimesTen and other early in-memory DBMSs during 1990s
    \item Breath-First Search ordering of keys
\end{itemize}
\subsection*{Advantages}
    \begin{itemize}
        \item Uses less memory because it does not store keys inside of each node
        \item Inner nodes contain key/value pairs (like B-Tree)
    \end{itemize}
\subsection*{Disadvantages}
    \begin{itemize}
        \item Difficult to rebalance
        \item Difficult to implement safe concurrent access
        \item Have to chase pointers when scanning range or performing binary search insde of a node
    \end{itemize}

\section{Skip List~\cite{p668-pugh}}
\begin{itemize}
    \item Multiple levels of linked lists with extra pointers that \textbf{skip} over intermediate nodes
    \item Maintains keys in sorted order without requiring global rebalancing
    \item \textbf{Probibalistic data structure}
    \item A collection of lists at different levels
    \begin{itemize}
        \item Lowest level is a sorted, singly linked list of all keys
        \item 2nd level links every other key
        \item 3rd level links every fourth key
        \item In general, a level has half the keys of one below it
    \end{itemize}
    \item To insert a new key, flip a coin to decide how many levels to add the new key into
    \item \textbf{Provides approximate O(log n) search times}
    \item Advantages
    \begin{itemize}
        \item Uses less memory than a B+Tree (only if you don't include reverse pointers)
        \item Insertions and deletions do not require rebalancing
        \item It is possible to implemment a concurrent skip list using only CAS instructions
    \end{itemize}
\end{itemize}

\section{Concurrent Skip List~\cite{hpugh-concurrent-tr1990}}
\begin{itemize}
    \item Can implement insert and delete without locks using only CaS operations
    \item The data structure only support links in one direction because CaS can only swap on pointer atomically
    \item For insert, CaS should happen \textbf{bottom up}
    \item If the DBMS invokes operation on the index, it can never "fail"
    \item A txn can only abort due to higher-level conflicts
    \item If a CaS fails, then the index will retry until it succeeds
\end{itemize}
\subsection*{\textbf{Delete}}
    \begin{itemize}
        \item First \textbf{logically} remove a key from the index by setting a flag to tell threads to ignore
        \item Then \textbf{physically} remove the key once we know that no other thread is holding the reference
        \item Deletion should happen \textbf{top to bottom}
    \end{itemize}

\subsection*{Skip List Optimizations~\cite{skiplists-done-right2016}}
    \begin{itemize}
        \item Reducing RAND() invocations
        \item Packing multiple keys in a node
        \item Reverse iteration with a stack
        \item Reusing nodes with memory pools
    \end{itemize}
\subsection*{\textbf{Combined Nodes}}
    \begin{itemize}
        \item Store multiple keys in a single node
        \item Insert Key: Find the node where it should go and look for a free slot. Perform CaS to store new key. If no slot is available, insert new node
        \item Search Key: Perform linear search on keys in each node
    \end{itemize}
\subsection*{\textbf{Reverse Search}}
    \begin{itemize}
        \item Perform a regular range query, but add the keys into a stack
        \item In the end, pop they keys from the stack, and they will pop in reverse order
    \end{itemize}

\section{Bw-Tree~\cite{bwtree-icde2013}}
\begin{itemize}
    \item Because CaS only updates a single address at a time, this limits the design of our data structure
    \item Bw-Tree is a latch free B+Tree index
    \item Threads never need to set latches or block
\end{itemize}
\subsection*{\textbf{Deltas}}
    \begin{itemize}
        \item No updates in place
        \item Reduces cache invalidation
        \item Each update to a page produces a new delta
        \item Delta physically points to base page
        \item Install delta adress in physical address slot of mapping table using CaS
    \end{itemize}

\subsection*{\textbf{Mapping Table}}
    \begin{itemize}
        \item Allows for CaS of physical locations of pages
        \item Maps page ID's to physical address locations
    \end{itemize}
\subsection*{Delta Updates}
    \begin{itemize}
        \item Each update to a new page produces a new delta
        \item Delta physically points to base page
        \item Install delta address in physical address slot of mapping table using CAS
    \end{itemize}
\subsection*{Search}
    \begin{itemize}
        \item Traverse tree like regular B+ tree
        \item If mapping table points to delta chain, stop at first occurence of search key
        \item Otherwise, perform binary search on base page
    \end{itemize}
\subsection*{Contention updates}
    \begin{itemize}
        \item Threads may try to install updates to same state of the page
        \item Winner succeeds, any loser must retry or abort
    \end{itemize}
\subsection*{Delta Types}
    \begin{itemize}
        \item Record Update Deltas: Insert/update/deletes
        \item Structure Modification Deltas: Split/Merge information
        \item Consolidation
        \begin{itemize}
            \item Consolidate updates by creating new page with deltas applied
            \item CaS-ing the mapping table address ensures no deltas are missed
        \end{itemize}
    \end{itemize}
\subsection*{\textbf{Garbage Collection}}
    \begin{itemize}
        \item Operations are tagged with an \textbf{epoch}. Also called RCU in Linux
        \item Each epoch tracks the threads that are part of it and the objects that can be reclaimed
        \item Thread joins an epoch prior to each operation and post objects that can be reclaimed for the current epoch (not necessarily the one it joined)
        \item Garbage for an epoch is reclaimed only when all threads have exited the epoch
    \end{itemize}



% ==================================================================
% BIBLIOGRAPHY
% ==================================================================
\newpage
\bibliographystyle{abbrvnat}
\bibliography{08-oltpindexes1}








\end{document}
