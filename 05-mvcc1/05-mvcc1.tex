\documentclass[11pt]{article}

\newcommand{\lecturenumber}{05}
\newcommand{\lecturename}{Multi-Version Concurrency Control (MVCC) -- Part 1}
\newcommand{\lecturedata}{2018-01-31}

\usepackage{../dbnotes}

\begin{document}

\maketitle
\thispagestyle{plain}

%% ==================================================================
%% COMPARE AND SWAP
%% ==================================================================
\section{Compare-and-Swap}
\begin{itemize}
    \item
    Atomic instruction that compares contents of a memory location $M$ to a given value $V$.
    If the values are equal, installs new given value $V'$ in $M$. Otherwise, operation fails.
    
    \item
    Can be done with an \textbf{intrinsic} function call, a function that a compiler knows to 
    convert to an instruction without having to write the machine code.
\end{itemize}

%% ==================================================================
%% ISOLATION LEVELS
%% ==================================================================
\section{Isolation Levels}
\begin{itemize}
    \item
    Serializability is useful because it allows programmers to ignore concurrency issues but 
    enforcing it may allow too little parallelism and limit performance.
    
    \item
    We may want to use a weaker level of consistency to improve scalability.
    
    \item 
    Isolation levels control the extent that a transaction is exposed to the actions of other 
    concurrent transactions.
    
    \item Anomalies:
    \begin{itemize}
        \item \textbf{Dirty Read:}
        Reading uncommitted data.
        
        \item \textbf{Unrepeatable Reads:}
        Redoing a read results in a different result.
        
        \item \textbf{Phantom Reads:}
        Insertion or deletions result in different results for 
        the same range scan queries.
    \end{itemize}
    
    \item Isolation levels (strongest to weakest)
    \begin{enumerate}
        \item \isoLevel{Serializable}:
        No Phantoms, all reads repeatable, and no dirty reads.
        
        \item \isoLevel{Repeatable Reads}:
        Phantoms may happen.
        
        \item \isoLevel{Read-Committed}:
        Phantoms and unrepeatable reads may happen.
        
        \item \isoLevel{Read-Uncommitted}:
        All anomalies may happen.
    \end{enumerate}
    
    \item
    The isolation levels defined as part of SQL-92 standard only focused on anomalies that can 
    occur in a 2PL-based DBMS~\cite{p1-berenson}. There are two additional isolation levels:
    \begin{enumerate}
        \item \isoLevel{Cursor Stability}
        \begin{itemize}
            \item Between repeatable reads and read committed
            \item Prevents Lost Update Anomaly.
            \item Default isolation level in \dbSys{IBM DB2}.
        \end{itemize}
        
        \item \isoLevel{Snapshot Isolation}
        \begin{itemize}
            \item
            Guarantees that all reads made in a transaction see a consistent snapshot of the 
            database that existed at the time the transaction started.
            
            \item
            A transaction will commit only if its writes do not conflict with any concurrent 
            updates made since that snapshot.
            
            \item
            Susceptible to write skew anomaly.
        \end{itemize}
    \end{enumerate}
\end{itemize}

%% ==================================================================
%% MULTI-VERSION CONCURRENCY CONTROL
%% ==================================================================
\section{Multi-version Concurrency Control (MVCC)}
Originally proposed in 1978 MIT dissertation~\cite{reed1978naming}.

MVCC is currently the best approach for supporting transactions in mixed workloads.
The DBMS maintains multiple \textbf{physical} versions of an object of a single 
\textbf{logical} object in the database. 
When a transaction writes to an object, the DBMS, creates a new version of that object.
When a transaction reads an object, it reads the newest version that existed when a 
transaction started.
    
\textbf{Main Benefits}
\begin{itemize}
    \item Writes do not block readers.
    \item Read-only transactions can read a consistent snapshot without acquiring locks.
    \item Easily support time-travel queries.
\end{itemize}

MVCC is more than just a ``concurrency control protocol''. It completely affects how the 
DBMS manages transactions and the database. There are four key design decisions:~\cite{p781-wu}
\begin{itemize}
    \item Concurrency Control Protocol
    \item Version Storage
    \item Garbage Collection
    \item Index Management
    \item Transaction Id Wraparound
\end{itemize}
% \item Tuple Header Format
% \begin{itemize}
%     \item Txn-ID: Unique Txn Identifier
%     \item Begin/End Timestamp: Version lifetime
%     \item Pointer: Points to next or previous tuple in version chain
%     \item Additional Meta-Data
% \end{itemize}

%% ==================================================================
%% CONCURRENCY CONTROL PROTOCOL
%% ==================================================================
\section{Concurrency Control Protocol}
The DBMS is able to use all of the same concurrency control protocols under MVCC.
% \begin{itemize}
%     \item Approaches
%     \begin{enumerate}
%         \item Timestamp Ordering
%         \item OCC
%         \item Two-Phase Locking
%     \end{enumerate}

\textbf{Timestamp Ordering (MV-TO):}
\begin{itemize}
    \item
    Use a \texttt{read-ts} field in the header to keep track of the timestamp of the last 
    transaction that read it.
    
    \item
    Transaction is allowed to read version if the lock is unset and its $T_id$ is between 
    \texttt{begin-ts} and \texttt{end-ts}.
    
    \item
    For writes, transaction creates a new version if no other transaction holds lock and 
    $T_id$ is greater than \texttt{read-ts}.
\end{itemize}

%% ==================================================================
%% VERSION STORAGE
%% ==================================================================
\section{Version Storage}
The DBMS uses the tuple's pointer field to create a latch-free \textbf{version chain} per logical 
tuple. This allows the DBMS to find the version that is visible to a particular transaction at 
runtime. Indexes always point to ``head'' of the chain.

Thread store versions in ``local'' memory regions to avoid contention on centralized data 
structures. Different storage schemes determine where/what to store for each version.

For non-inline attributes, the DBMS can reuse pointers to variable-length pool for values that do 
not change between versions. This requires reference counters to know when it is safe to free 
memory. This optimization also makes it more difficult to relocate memory from the variable-length 
pool.
    
%% ----------------------------------------------------
%% Append-Only Storage
%% ----------------------------------------------------
\subsection*{Append-Only Storage}
    \begin{itemize}
        \item
        All of the physical versions of a logical tuple are stored in the same table space 
        (table heap).
        
        \item
        On update, append new tuple to same table heap in an empty slot.
        
        \item \textbf{Oldest-to-Newest (O2N):}
        Append new version to end of chain, traverse entire chain on lookup.
        
        \item \textbf{Newest-to-Oldest (N2O):}
        Have to update index pointers for every new version, but do not have to traverse chain on 
        look ups.
\end{itemize}

%% ----------------------------------------------------
%% Time-Travel Storage
%% ----------------------------------------------------
\subsection*{Time-Travel Storage}
\begin{itemize}
    \item
    On every update, copy current version to the time-travel table, and update pointer.
    
    \item
    Overwrite master version in main table, and update pointer.
\end{itemize}

%% ----------------------------------------------------
%% Delta Storage
%% ----------------------------------------------------
\subsection*{Delta Storage}
\begin{itemize}
    \item
    On every update, copy only the values that were modified into the delta storage and 
    overwrite the master version.
    
    \item
    Transaction can recreate old versions by applying the delta in reverse order.
\end{itemize}

%% ==================================================================
%% GARBAGE COLLECTION
%% ==================================================================
\section{Garbage Collection}
The DBMS needs to remove \textbf{reclaimable} physical versions from the database over 
time. A version is reclaimable if (1) no active transaction in the DBMS can see that version or 
(2) the version was created by an aborted transaction.

%% ----------------------------------------------------
%% Tuple Level
%% ----------------------------------------------------
\subsection*{Tuple Level}
\begin{itemize}
    \item
    Find old versions by examining tuples directly.
    
    \item \textbf{Background Vacuuming:}
    Separate threads periodically scan the table and look for 
    reclaimable versions. Works with any version storage technique.
    
    \item \textbf{Cooperative Cleaning:}
    Worker threads identify reclaimable versions as they 
    traverse version change. Only works with O2N version chains.
\end{itemize}

%% ----------------------------------------------------
%% Transaction Level
%% ----------------------------------------------------
\subsection*{Transaction Level}
\begin{itemize}
    \item
    Transactions keep track of their old version so the DBMS does not have to scan tuples 
    to determine visibility.
    
    \item
    The DBMS determines when all versions created by a finishing transaction are no 
    longer visible.
    
    \item
    May still maintain multiple threads to reclaim the memory fast enough for the workload.
\end{itemize}

%% ==================================================================
%% INDEX MANAGEMENT
%% ==================================================================
\section{Index Management}
How often the DBMS updates index depends on whether system creates new versions when a 
tuple is updated.

%% ----------------------------------------------------
%% Primary Key
%% ----------------------------------------------------
\subsection*{Primary Key}
Primary key indexes always point to the version chain head.
If a transaction updates a primary key attribute(s), then this is treated as a \texttt{DELETE} 
followed by an \texttt{INSERT}.

%% ----------------------------------------------------
%% Secondary Indexes
%% ----------------------------------------------------
\subsection*{Secondary Indexes}
        
\textbf{Approach \#1: Logical Pointer}
\begin{itemize}
    \item Use a fixed identifier per tuple that \textbf{does not change}
    \item Requires an extra indirection layer
    \item Secondary indexes can store Primary Key or Tuple ID
\end{itemize}

\textbf{Approach \#2: Physical Address}
\begin{itemize}
    \item Pointer physical points to tuple
\end{itemize}

%% ==================================================================
%% Transaction Id Wraparound
%% ==================================================================
\section{Transaction Id Wraparound}
If the DBMS reaches the max value for its timestamps, it will have to wrap around and 
start at zero. This will make all previous versions be in the ``future'' from new transactions.

\textbf{Postgres \texttt{Txn-ID} Wraparound}
\begin{itemize}
    \item
    Stop accepting new commands when the system gets close to the max transaction id.
        
    \item
    Set a flag in each tuple header that says that it is ``frozen'' in the past. Any new 
    transaction id will always be newer than a frozen version.
        
    \item
    Runs the vacuum before the system gets close to this upper limit.
\end{itemize}

% ==================================================================
% BIBLIOGRAPHY
% ==================================================================
\newpage
\bibliographystyle{abbrvnat}
\bibliography{05-mvcc1}

\end{document}
