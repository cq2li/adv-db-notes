\documentclass[11pt]{article}

\newcommand{\lecturenumber}{04}
\newcommand{\lecturename}{Multi-Version Concurrency Control (Protocols)}
\newcommand{\lecturedata}{2020-01-27}

\newcommand{\mvccField}[1]{\texttt{#1}\xspace}

\usepackage{../dbnotes}

\begin{document}

\maketitle
\thispagestyle{plain}

%% ==================================================================
%% MICROSOFT HEKATON
%% ==================================================================
\section{\dbSys{Microsoft Hekaton}}
Hekaton started as an internal at Microsoft in 2008. The goal was to create new in-memory OLTP 
engine for \dbSys{Microsoft SQL Server} (MSSQL). The project was led by database experts Paul 
Larson and Mike Zwilling. It was important that the new 
engine integrated with \dbSys{MSSQL} ecosystem. It was also necessary that the engine support all 
possible OLTP workloads with predictable performance.

Key Lessons from Hekaton on building a high-performance OLTP DBMS:
\begin{itemize}
    \item
    Use only lock-free data structures. This means no latches, spin locks, or critical 
    sections for indexes, transaction map, memory allocator, garbage collector.
    Not necessarily true for indexes; systems using traditional latching methods for indexing more 
    often than not outperform lock-free implementations.
        
    \item
    Only one single serialization point in the DBMS to get the transaction's begin and commit 
    timestamp using an atomic addition (i.e., compare-and-swap -- CaS). At high parallelism levels 
    this could be a bottleneck. In this case we can improve performance with batching 
    techniques.
\end{itemize}

The Hekaton concurrency control protocol relies on CaS to install new 
versions~\cite{p298-larson}.
\begin{itemize}
    \item
    Each transaction is assigned a timestamp when they begin (\mvccField{BeginTS}) and when they 
    commit (\mvccField{EndTS}). Hekaton uses a special bit for transaction's \mvccField{BeginTS} 
    field to indicate whether it represents an uncommitted transaction.
        
    \item
    Each tuple contains two timestamps that represents their visibility and current state.
    \begin{itemize}
        \item \textbf{\texttt{BEGIN-TS}}:
        The \mvccField{BeginTS} of the active transaction or the \mvccField{EndTS} of the committed 
        transaction that created it.
            
        \item \textbf{\mvccField{END-TS}}:
        The \mvccField{BeginTS} of the active transaction that created the next version or 
        infinity or the \mvccField{EndTS} of the committed transaction that created it
    \end{itemize}
\end{itemize}

% \textbf{Speculative Reads}:
Hekaton allows transactions to speculatively read versions of transactions of uncommitted 
transactions. Then checks in validations if the transactions that it read uncommitted data 
committed. The DBMS does not allow speculative write; first transaction to write a new version 
succeeds, subsequent transactions that attempt to write to the same tuple are aborted.

%% ----------------------------------------------------
%% Transaction State Map
%% ----------------------------------------------------
\subsection*{Transaction State Map}
Global map of all transactions states in the system. Transactions have to consult this map to 
determine the state of a transaction.
\begin{itemize}
    \item \texttt{ACTIVE}:
    The transaction is executing read/write operations.
    
    \item \texttt{VALIDATING}:
    The transaction has invoked commit and the DBMS is checking whether it is valid.
    
    \item \texttt{COMMITTED}:
    The transaction is finished, but may not have updated its version's timestamps.
    
    \item \texttt{TERMINATED}:
    The transaction has updated the timestamps for all of the versions that it created.
\end{itemize}

%% ----------------------------------------------------
%% Transaction Life-Cycle
%% ----------------------------------------------------
\subsection*{Transaction Life-Cycle}
\begin{enumerate}
    \item \textbf{Begin}:
    Get \mvccField{BeginTS}, set state to ACTIVE. The transaction then begins its normal 
    execution for read/write queries. In addition to updating version information, the DBMS
    also tracks each transaction's read set, scan set and write set.
    
    \item \textbf{Pre-Commit}:
    Assign the transaction its \mvccField{EndTS} and set its state in the transaction map to 
    \texttt{VALIDATING}. During validation the DBMS checks whether the transaction's reads and 
    scans produce the same result (to ensure serializability). If validation passes, then the DBMS 
    writes new versions to redo log.
    
    \item \textbf{Post-Processing}:
    Update the timestamps for the versions that the transaction either created or 
    invalidated. The transaction updates the \mvccField{BeginTS} in new versions, \texttt{CommitTS} 
    in old versions.
    
    \item \textbf{Terminate}:
    Set transaction state to \text{TERMINATED} and remove from map.
\end{enumerate}

%% ----------------------------------------------------
%% Transaction Meta-Data
%% ----------------------------------------------------
\subsection*{Transaction Meta-Data}
The DBMS maintains additional meta-data about what transactions did during execution.

\begin{itemize}
    \item \textbf{Read Set}:
    Pointers to every version that the transaction read.
    
    \item \textbf{Write Set:}
    Pointers to the versions that the transaction updated (old and new), deleted (old), 
    and inserted (new).
    
    \item \textbf{Scan Set:}
    Stores enough information needed to perform each scan operation.
    
    \item \textbf{Commit Dependencies:}
    List of transactions that are waiting for this transaction to finish.
\end{itemize}

%% ----------------------------------------------------
%% Transaction Validation
%% ----------------------------------------------------
\subsection*{Transaction Validation}
Hekaton used both optimistic / pessimistic schemes for transactions. Optimistic schemes outperforms 
pessimistic ones as the number of cores and parallelism increase.

\begin{itemize}
    \item \textbf{Optimistic Transactions}:
    Check whether a version read is still visible at the 
    end of a transaction. Repeat all index scans to check for phantoms. 

    \item
    \textbf{Pessimistic Transactions}:
    Use shared and exclusive locks on records and buckets. Do 
    not need any validations and use a separate background thread to perform deadlock detection. 
    Under this setting, the DBMS does not need to perform validation.
\end{itemize}
    
The DBMS needs to scan transactions to provide its transactional guarantees.
\begin{itemize}
    \item \textbf{First-Writer Wins:}
    The version vector always points to the last committed version. Do not need to check whether 
    write-sets overlap.

    \item \textbf{Read Stability:}
    Check that each version read is still visible as of the end of 
    the transaction.
    
    \item \textbf{Phantom Avoidance:}
    Repeat each scan to check whether new versions have become visible since the transaction began.
    
    \item Extend of validation depends on isolation level:
    \begin{itemize}
        \item \isoLevel{SERIALIZABLE}: Read Stability + Phantom Avoidance.
        \item \isoLevel{REPEATABLE READS}: Read Stability.
        \item \isoLevel{SNAPSHOT ISOLATION}: None
        \item \isoLevel{READ COMMITTED}: None
    \end{itemize}
\end{itemize}

%% ==================================================================
%% TUM HYPER
%% ==================================================================
\section{\dbSys{HyPer}}
Read/scan set validations are expensive if transactions access a lot of data. Appending new versions 
hurts the performance of OLAP scans due to pointer chasing and branching. Record level conflict 
checks may be too coarse-grained and incur false positives.

\dbSys{HyPer} uses a MVCC implementation that is designed for HTAP workloads~\cite{p677-neumann}. 
It is a column-store with delta record versioning. Avoids write-write conflicts by aborting 
transactions that try to update an uncommitted object.
\begin{itemize}
    \item In-Place updates for non-indexed attributes
    \item Delete/Insert updates for indexed attributes
    \item N2O version chains
    \item No Predicate Locks and No Scan Checks
\end{itemize}

%% ----------------------------------------------------
%% Transaction Validation
%% ----------------------------------------------------
\subsection*{Transaction Validation}
Like \dbSys{Hekaton}, \dbSys{HyPer} uses the first-writer wins policy.
Although \dbSys{HyPer}'s MVCC protocol supports multiple writers, the actual implementation uses 
single-threaded execution for write transactions.

It also uses a technique called \textit{precision locking} to enforce 
serializability~\cite{jordan81}. The system only stores each transaction's read predicates 
and not its entire read set.
For each validating transactions, the DBMS evaluates its \sql{WHERE} clause based on each delta 
record of any transactions that committed \textit{after} the committing transaction has started.
\begin{itemize}
    \item
    If the predicate evaluates to false, then the already committed transaction did not create 
    a version that the committing transaction should have read.
        
    \item
    If the predicate evaluates to true, then this means that the committing transaction should have 
    read the version created by the other transaction. Thus, the DBMS will abort our validating     
    transactions and rollback its changes.
\end{itemize}

%% ----------------------------------------------------
%% Version Synopsis
%% ----------------------------------------------------
\subsection*{Version Synopsis}
Checking the version vector for every tuple during long table scans is expensive. This is wasted 
work for cold data that is unlikely to have previous versions. Thus, to avoid this work, 
\dbSys{HyPer} maintains special \textit{version synopsis} tags per block to keep track of ranges of 
tuples that do not have versions. This is a separate column that tracks the position of the first 
and last versioned tuple in a block of tuples.
When scanning tuples, the DBMS can check for strides of tuples without older versions and 
execute more efficiently.

%% ==================================================================
%% SAP HANA
%% ==================================================================
\section{\dbSys{SAP HANA}}
\dbSys{SAP HANA} is an in-memory HTAP DBMS with time-travel version storage (N2O).~\cite{farber2012sap}.
\begin{itemize}
    \item Supports both optimistic and pessimistic MVCC.
    \item Latest versions are stored in time-travel space.
    \item Hybrid storage layout (row + columnar).
    \item Based on \dbSys{P*TIME}, \dbSys{TREX}, and \dbSys{MaxDB}.
\end{itemize}

%% ----------------------------------------------------
%% Version Storage
%% ----------------------------------------------------
\subsection*{Version Storage}
The DBMS stores the oldest version in the main data table.
Each tuple in the main data table contains a flag to denote whether there exists newer versions in 
the version space (i.e., time-travel table). For the time-travel table, the DBMS maintains a 
separate hash table that maps the logical record id for the tuple to the head of the version chain. 
This avoids storing additional metadata about versioning in the main table.

Instead of embedding begin/end timestamps in each tuple, \dbSys{HANA} instead stores a 
pointer in each tuple to a context object. This object contains all the meta-data about the 
transaction that created a version with the data. This allows the DBMS to update the timestamps for 
multiple just by writing to this context instead of having to update each tuple individually.

The system has another layer of indirection for the meta-data about whether a transaction has 
committed. This separate context object keeps track of when transactions commit and when their log 
records have been flushed to disk.

%% ==================================================================
%% CICADA
%% ==================================================================
\section{\dbSys{Cicada}}
\dbSys{Cicada} is an academic in-memory OLTP engine based on optimistic MVCC with append-only 
storage (N2O)~\cite{p21-lim}. It was designed with specific optimizations for supporting both low 
and high contention workloads:
\begin{itemize}
    \item Best-effort in-lining
    \item Loosely Synchronized Clocks
    \item Contention Aware validation
    \item Index Nodes Stored in Tables
\end{itemize}

%% ----------------------------------------------------
%% Best-Effort Inlining
%% ----------------------------------------------------
\subsection*{Best-Effort In-lining}
\begin{itemize}
    \item
    Record meta-data is stored in a fixed location.
    
    \item
    Threads will attempt to inline read-mostly version within this meta-data to reduce 
    version chain traversals.
\end{itemize}

%% ----------------------------------------------------
%% Transaction Validation
%% ----------------------------------------------------
\subsection*{Transaction Validation}

% TODO: Add intro paragraph to say that Cicada has several optimizations to speed up the validation 
% step in their DBMS.

\begin{itemize}
    \item \textbf{Contention-aware Validation:}
    Validate access to recently modified (high contention) records first. Instead of comparing 
    transactions' read sets, \dbSys{Cicada} tries to reorder them based on their write timestamps. 
    The DBMS keeps track of what transactions are mostly likely to conflict for a 
    transaction attempting to commit, so it then checks for conflicts with them first.
    
    \item \textbf{Early Consistency Check:}
    Pre-validated access set before making global writes. Rather than waiting until the end of 
    a transaction to perform validation, we can preemptively kill transactions that might abort 
    later on.
      
    \item \textbf{Incremental Version Search:}
    Resume from last search location in version list to improve cache performance.
\end{itemize}

If the DBMS knows that most of the recently executed transactions committed successfully,
then it can skip Contention-Aware Validation and Early Consistency check


% ==================================================================
% BIBLIOGRAPHY
% ==================================================================
\newpage
\bibliographystyle{abbrvnat}
\bibliography{04-mvcc2}

\end{document}
