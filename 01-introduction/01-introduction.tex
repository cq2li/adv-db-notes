\documentclass[11pt]{article}

\newcommand{\lecturenumber}{01}
\newcommand{\lecturename}{Course Introduction and History of DBMSs}
\newcommand{\lecturedata}{2018-01-17}

\usepackage{../dbnotes}

\begin{document}

\maketitle
\thispagestyle{plain}

% \section{Course Objectives}
% \begin{itemize}
%     \item Learn about modern practices in database internals and systems programming
%     \item Students will become proficient in
%     \begin{itemize}
%         \item Writing correct + performant code
%         \item Proper documentation + testing
%         \item Code reviews
%         \item Working on a large code base
%     \end{itemize}
%     \item Course will be focused on only \textbf{single node} systems
%     \item Will cover state-of-the-art topics, not classic DBMSs
% \end{itemize}

\section{Database History Observations}
\begin{itemize}
    \item
    Stonebraker argues that a lot of the issues relevant a long time ago are still relevant 
    today~\cite{stonebraker2005goes}.
    
    \item
    The ``SQL vs. NoSQL'' debate of the 2010s is reminiscent of the ``Relational vs. 
    CODASYL'' debate from the 1970s~\cite{michaels76}.
\end{itemize}

\section{The 1960s}
\begin{itemize}
    \item \textbf{First DBMS - Integrated Data System (IDS)}~\cite{bachman-interview,haigh16}
    \begin{itemize}
        \item
        Developed internally at GE in the early 1960s by Charles Bachman.
        
        \item
        GE sold their computing division to Honeywell in 1969.
        
        \item
        Based on the \textit{Network Data Model} that supported 
        \textit{Tuple-at-a-time} query execution~\cite{bachman66}.
    \end{itemize}
    
    \item \textbf{CODASYL}~\cite{taylor76}
    \begin{itemize}
        \item
        Proponents of COBOL and the network data model people got together and proposed a standard 
        for how programs will access a database.
        
        \item
        \textit{Network Data Model} made complex queries difficult and was easily corruptible.
    \end{itemize}
    
    \item \textbf{IBM Information Management System (IMS)}~\cite{klein12}
    \begin{itemize}
        \item
        Early database system developed to keep track of purchase orders for Apollo moon mission.
        
        \item
        Based on the \textit{Hierarchical Data Model} that organized collections of data with
        parent/child relationships.
        
        \item
        Programmer-defined physical storage format (e.g., hash table vs. tree) and tuple-at-a-time 
        query execution.
    \end{itemize}
\end{itemize}

\section{The 1970s}
\begin{itemize}
    \item \textbf{Relational Model}~\cite{codd70}
    \begin{itemize}
        \item
        Ted Codd was a mathematician working at IBM Research who saw developers constantly 
        changing their codebase whenever the database's schema changed.
        
        \item
        Codd created the relational model abstraction to avoid this maintenance based on three key 
        ideas:
        \begin{enumerate}
            \item Store database in simple data structures
            \item Access data through high-level language
            \item Physical storage left up to implementation
        \end{enumerate}
    \end{itemize}
    \item Early implementations of relational DBMSs:
    \begin{itemize}
        \item IBM Research: \dbSys{System R}
        \item U.C. Berkeley (Mike Stonebraker): \dbSys{INGRES}
        \item Relational Software, Inc. (Larry Ellison): \dbSys{Oracle}
    \end{itemize}
\end{itemize}

\section{The 1980s}
\begin{itemize}
    \item The relational model wins the database marketplace over CODASYL.
    \begin{itemize}
        \item
        IBM released their first relational DBMS (\dbSys{DB2}) in 1983.
        
        \item
        System R's ``SEQUEL'' declarative query language becomes the standard (later renamed to 
        ``SQL'').
        
        \item
        Many new enterprise DBMSs are invented (\dbSys{Informix}, \dbSys{Sybase}, 
        \dbSys{TeraData}) but Oracle wins marketplace.
        
        \item
        Stonebraker leaves INGRES, returns to Berkeley and starts the Postgres project.
    \end{itemize}
    
    \item \textbf{Object-Oriented Databases}~\cite{zdonik89}
    \begin{itemize}
        \item
        Argued that how people wrote code and how data is stored in a database is different
        
        \item
        Avoid ``relational object impedance mismatch'' by tightly coupling objects and database.
        
        \item
        Few of these original DBMSs from the 1980s still exist today. These systems performed 
        poorly when executing complex queries. There was also no standard API or programming 
        language.
    \end{itemize}
\end{itemize}

\section{The 1990s}
\begin{itemize}
    \item No major advancements in database systems or application workloads.
    \item Major Events:
    \begin{itemize}
        \item
        Microsoft forks \dbSys{Sybase} and creates \dbSys{SQL Server} for Windows NT.
        
        \item
        \dbSys{MySQL} is written as a replacement for \dbSys{mSQL}.
        
        \item
        Illustra (the commercial version of \dbSys{Postgres}) gets bought by \dbSys{Informix}. 
        Graduate students at Berkeley take the original academic \dbSys{Postgres} code and adds 
        support for SQL.
        
        \item
        \dbSys{SQLite} started in early 2000.
    \end{itemize}
\end{itemize}

\section{The 2000s}
\begin{itemize}
    \item \textbf{Data Warehouses}
    \begin{itemize}
        \item Distributed / Shared-Nothing
        \item Relational / SQL
        \item Usually closed-source
        \item
        Significant performance benefits from using \textit{Decomposition Storage Model}
        (i.e., columnar storage).
    \end{itemize}
    
    \item \textbf{NoSQL Systems}~\cite{cattell10}
    \begin{itemize}
        \item Focus on high-availability and high-scalability
        \item Schema-less
        \item Non-relational data models
        \item No ACID transactions
        \item Custom APIs instead of SQL
        \item Usually open-source
    \end{itemize}
\end{itemize}

\section{The 2010s}
\begin{itemize}
    \item
    \textbf{NewSQL Systems}~\cite{aslett11,pavlo16}
    \begin{itemize}
        \item
        Strive to provide same scalability and performance for OLTP workloads as NoSQL DBMSs 
        without giving up ACID.
        \item Relational / SQL
        \item Distributed
        \item Usually closed-source
    \end{itemize}
    
    \item \textbf{Hybrid Transactional-Analytical Processing (HTAP)}~\cite{pezzini14}
    \begin{itemize}
        \item
        Execute fast OLTP like a NewSQL system while also executing complex OLAP queries like 
        a data warehouse system.
        \item Distributed / Shared-Nothing
        \item Relational/SQL
        \item Mixed open/closed-source
    \end{itemize}
    
    \item \textbf{Cloud-based Database Systems}
    \begin{itemize}
        \item
        First database-as-a-service (DBaaS) offerings were ``containerized'' versions of existing 
        DBMSs (e.g., Amazon RDS).
        
        \item
        There are new DBMSs that are designed from scratch explicitly for running in a cloud 
        environment.
    \end{itemize}
    
    \item \textbf{Specialized Database Systems}
    \begin{itemize}
        \item Shared Disk (HDFS, EBS)
        \item Embedded
        \item Time-Series
        \item Multi model
        \item Blockchain
    \end{itemize}
\end{itemize}

% ==================================================================
% BIBLIOGRAPHY
% ==================================================================
\newpage
\bibliographystyle{abbrvnat}
\bibliography{01-introduction}

\end{document}
