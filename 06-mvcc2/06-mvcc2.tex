\documentclass[11pt]{article}

\newcommand{\lecturenumber}{6}
\newcommand{\lecturename}{Multi-Version Concurrency Control (MVCC) - Part 2}
\newcommand{\lecturedata}{2018-02-05}

\usepackage{../dbnotes}

\begin{document}

\maketitle
\thispagestyle{plain}

\section{\dbSys{Microsoft Hekaton}}
\begin{itemize}
    \item Icubator project started in 2008 to create new OLTP engine for \dbSys{MSFT SQL Server}
    \item Led by database ballers Paul Larson and Mike Zwilling
    \item Had to integrate with \dbSys{MSSQL} ecosystem
    \item Had to support all possible OLTP workloads with predicatable performance
\end{itemize}
    \subsection*{\dbSys{Hekaton} MVCC~\cite{p298-larson}}
    \begin{itemize}
        \item Each txn is assigned a timestamp when they begin (BeginTS) and when they commit (EndTS)
        \item Each tuple contains two timestamps that represents their visibility and current state
        \begin{itemize}
            \item \textbf{BEGIN}: The BeginTS of the active txn or the EndTS of the committed txn that created it
            \item \textbf{END}: The BeginTS of the active txn that created the next version or infinity or the EndTS of the committed txn that created it

        \end{itemize}
        \item Speculative Reads: Hekaton allows txns to read versions of transactions of uncommitted txns. Then checks in validations if the txns that it read uncommitted data committed
        \item Does not allow speculative write: First txn to write lives, second gets aborted
    \end{itemize}

    \subsection*{Transaction State Map}
    \begin{itemize}
        \item Global map of all txns states in the system
        \begin{itemize}
            \item ACTIVE: The txn is executing read/write operations
            \item VALIDATING: The txn has invoked commit and the DBMS is checking whether it is valid
            \item COMMITTED: The txn is finished, but may not have updated its version's TS
            \item TERMINDATED: The txn has updated the TS for all of the versions that it created
        \end{itemize}
    \end{itemize}
    \subsection*{Transaction Meta-Data}
    \begin{itemize}
        \item Read Set: Pointers to every version read
        \item Write Set: Pointers to versions updated (old and new), versions deleted (old), and version inserted (new)
        \item Scan Set: Stores enough information needed to perform each scan operation
        \item Commit Dependencies: List of txns that are waiting for this txn to finish
    \end{itemize}
    \subsection*{Transaction Validation}
    \begin{itemize}
        \item Read Stability: Check that each version read is still visible as of the end of the txn
        \item Phantom Avoidance: Repeat each scan to check whether new versions have become visible since the txn began
        \item Extend of validation depends on isolation level
        \begin{itemize}
            \item \isoLevel{SERIALIZABLE}: Read Stability + Phantom Avoidance
            \item \isoLevel{REPEATABLE READS}: Read Stability
            \item \isoLevel{SNAPSHOT ISOLATION}: None
            \item \isoLevel{READ COMMITTED}: None
        \end{itemize}
    \end{itemize}
    \subsection*{Lessons}
    \begin{itemize}
        \item Use only lock-free data structures
        \begin{itemize}
            \item No latches, spin locks, or critical sections
            \item Indexes, txn map, memory alloc, garbage collector
        \end{itemize}
        \item Only one single serialization point in the DBMS to get the txn's begin and commit timestamp using an Atomic Addition (CAS)
    \end{itemize}

\section{\dbSys{HyPer}}
    \subsection*{Observations}
    \begin{itemize}
        \item Read/scan set validations are expensive if txns access a lot of data
        \item Appending new versions hurts the performance of OLAP scans due to pointer chasing and branching
        \item Record level conflict checks may be too coarse-grained and incur false positives
    \end{itemize}
    \subsection*{\dbSys{HyPer} MVCC~\cite{p677-neumann}}
    \begin{itemize}
        \item Designed for HTAP workloads
        \item Column-store with delta record versioning
        \begin{itemize}
            \item In-Place updates for non-indexed attributes
            \item Delete/Insert updates for indexed attributes
            \item N2O version chains
            \item No Predicate Locks and No Scan Checks
        \end{itemize}
        \item Avoids write-write conflicts by aborting txns that try to update an uncommitted object
        \item Designed for HTAP workloads
    \end{itemize}
    \subsection*{Validation}
    \begin{itemize}
        \item First-Writer Wins
        \begin{itemize}
            \item The version vector always points to the last committed version
            \item Do not need to check whether write-sets overlap
        \end{itemize}
        \item Check the undo buffers (i.e. delta records) of txns that committed after the validation txn started
        \item Compare the committed txn's write set for phandoms using \textbf{Precision Locking}
        \item Only need to store the txn's read predicates and not its entire read set
    \end{itemize}
    \subsection*{Version Synopsis}
    \begin{itemize}
        \item Store a separate column that tracks the position of the first and last versioned tuple in a block of tuples
        \item When scanning tuples, the DBMS can check for strides of tuples without older versions and execute more efficiently
    \end{itemize}


\section{\dbSys{CMU Cicada}}
\begin{itemize}
    \item In-memory OLTP engine based on optimistic MVCC with append-only storage (N2O) ~\cite{p21-lim}
    \begin{itemize}
        \item Best-effort inlining
        \item Loosely Synchronized Clocks
        \item Contention Aware validation
        \item Index Nodes Stored in Tables
    \end{itemize}
    \item Designed to be scalable for both low and high contention workloads
\end{itemize}

    \subsection*{Best-Effort Inlining}
    \begin{itemize}
        \item Record meta-data is stored in a fixed location
        \item Threads will attempt to inline read-mostly version within this meta-data to reduce version chain traversals
    \end{itemize}
    \subsection*{Fast Validation}
    \begin{itemize}
        \item Contention-aware Validation: Validate access to recently modified (high contention) records first
        \item Early Consistency Check: Pre-validated access set before making global writes
        \item If you know that recent txns committed succesfully, the skip Contention-Aware Validation and Early Consistency check
        \item Incremental Version Search: Resume from last search location in version list
\end{itemize}


% ==================================================================
% BIBLIOGRAPHY
% ==================================================================
\newpage
\bibliographystyle{abbrvnat}
\bibliography{06-mvcc2}



\end{document}
