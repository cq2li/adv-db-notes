\documentclass[11pt]{article}

\newcommand{\lecturenumber}{11}
\newcommand{\lecturename}{System Catalogs and Database Compression}
\newcommand{\lecturedata}{2018-02-21}

\usepackage{../dbnotes}

\begin{document}

\maketitle
\thispagestyle{plain}

%% ==================================================================
%% SYSTEM CATALOGS
%% ==================================================================
\section{System Catalogs}
Almost every DBMS stores their database catalog in itself. This requires specialized code for 
``bootstrapping'' catalog tables.
The meta-data are tuples stored in the database, but the DBMS code itself is written an imperative 
language. Thus, we need to wrap object abstraction around tuples.

The entire DBMS should be aware of transactions in order to automatically provide ACID 
guarantees for DDL commands and concurrent transactions.
Actions that modify the catalog should be treated like transactions to allow for ACID guarantees.

%% ----------------------------------------------------
%% Schema Changes
%% ----------------------------------------------------
\subsection{Schema Changes}

\textbf{Add Column:}
\begin{itemize}
    \item NSM:
    Copy tuples into new region in memory.
    
    \item DSM:
    Just create the new column segment.
\end{itemize}

\textbf{Drop Column:}
\begin{itemize}
    \item NSM \# 1: 
    Copy tuples into new region of memory.
    
    \item NSM \# 2: 
    Mark column as ``deprecated, clean up later.
    
    \item DSM:
    Drop the column and free memory.
\end{itemize}

\textbf{Change Column:}
\begin{itemize}
    \item
    Check whether the conversion is allowed to happen.
    
    \item
    Depends on default values.
\end{itemize}

\textbf{Create Index}
\begin{itemize}
    \item
    Scan the entire table and populate the index.
    Can use one or multiple threads. This is done in the context of a transaction.
    
    \item
    While the system is building the index, it also has to record changes made by transactions that 
    modified the table while another transaction was building the index.
    
    \item
    When the scan completes, the index building transaction locks the table and resolve changes 
that     were missed after the scan started. This ensures that there are no false negatives or 
false     positives.
\end{itemize}

\textbf{Drop Index}
\begin{itemize}
    \item Drop the index logically from the catalog
    \item It only becomes "invisible" when the transaction that dropped it commits
    \item All existing transactions will still have to update it
\end{itemize}

%% ----------------------------------------------------
%% Sequences
%% ----------------------------------------------------
\subsection{Sequences}
\begin{itemize}
    \item
    Typically stored in the catalog and used for maintaining a global counter.
    Also called ``auto-increment'' or ``serial'' keys.
    
    \item
    Sequences are not maintained with the same isolation protection as regular catalog 
    entries. This is because rolling back a transaction that incremented a sequence does not 
    rollback the change to that sequence.
\end{itemize}

%% ==================================================================
%% DATABASE COMPRESSION
%% ==================================================================
\section{Database Compression}
I/O is (almost) always the main bottleneck if the DBMS has to fetch data from disk. Thus, 
compression in these systems (almost) improves performance.

In-memory DBMSs are more complicated since they do not have to fetch data from disk to execute a 
query. But compressing the database reduces DRAM requirements and processing. They have to strike a 
balance between \textbf{speed} vs. \textbf{compression ratio}. In-memory DBMSs always choose 
speed.

There are key properties of real-world data sets that are amenable to compression:
\begin{itemize}
    \item
    Data sets tend to have highly \textbf{skewed} distributions for attribute values.
    
    \item
    Data sets tend to have high \textbf{correlation} between attributes of the same 
    tuple.
\end{itemize}

Given this, we want a database compression scheme to have the following properties:
\begin{itemize}
    \item
    Must produce fixed-length values.
    
    \item
    Allow the DBMS to postpone decompression as long as possible during query execution.
    
    \item
    Must be a \textbf{lossless} scheme because people do not like losing data. 
    Any kind of \textbf{lossy} compression has to be performed at the application level.
\end{itemize}
% 
%     \subsection{Zone Maps}
%     \begin{itemize}
%         \item Pre-computed aggregates for blocks of data
%         \item DBMS can check the zone map first to decide whether it wants to access the block
%     \end{itemize}

%% ----------------------------------------------------
%% Compression Granularity
%% ----------------------------------------------------
\subsection{Compression Granularity}
\begin{itemize}
    \item \textbf{Block Level:}
    Compress a block of tuples for the same table.
    
    \item \textbf{Tuple Level:}
    Compress the contents of the entire tuple (NSM only).
    
    \item \textbf{Attribute Level:}
    Compress a single attribute value within one tuple. This approach can target multiple 
attributes 
    for the same tuple. It is possible to 
    
    \item \textbf{Columnar Level:}
    Compress multiple values for one or more attributes stored for multiple tuples (DSM only).
\end{itemize}

%% ==================================================================
%% NAIVE COMPRESSION
%% ==================================================================
\section{Naive Compression}
Compress data using a general purpose algorithm (e.g., gzip). Scope of compression is only 
based on the data provided as input.

The data \textbf{must} be decompressed first before it can be read and (potentially) 
modified. This limits the scope of the compression scheme. These schemes also do not consider the 
high-level meaning or semantics of the data

There are two types of compression algorithms:
\begin{itemize}
    \item \textbf{Entropy Encoding:}
    More common sequences use less bits to encode, less common 
    sequences use more bits to encode. \\
    Example: Huffman Coding
    
    \item \textbf{Dictionary Encoding:}
    Build a data structure that maps data segments to an identifier. Replace those 
    segments in the original data with a reference to the segments position in the dictionary data 
    structure. \\
    Example: Most general purpose compression algorithms.
\end{itemize}

%% ==================================================================
%% ENCODING SCHEMES
%% ==================================================================
\section{Encoding Schemes}
The alternative to a naive compression scheme is one where the DBMS encodes the database itself in 
a compression format. The DBMS's query processing algorithms need to be aware of the data's 
encoding scheme in order to decipher it.


%% ----------------------------------------------------
%% Run-Length Encoding
%% ----------------------------------------------------
\subsection{Run-Length Encoding (RLE)}
Compress runs of the same value in a single column into triplets~\cite{p31-roth}.
\begin{itemize}
    \item The value of the attribute
    \item The start position in the column segment
    \item The \# of elements in the run
\end{itemize}
Requires the columns to be sorted intelligently to maximize compression opportunities.

%% ----------------------------------------------------
%% Bitmap Encoding
%% ----------------------------------------------------
\subsection{Bitmap Encoding}
Store a separate Bitmap for each unique value for a particular attribute where an offset 
in the vector corresponds to a tuple~\cite{franklin14}.
\begin{itemize}
    \item The $i^{th}$ position in the Bitmap corresponds to the $i^{th}$ tuple in the table
    \item Typically segmented into chunks to avoid allocating large blocks of contiguous memory
\end{itemize}
Only practical if the value cardinality is low.

%% ----------------------------------------------------
%% Delta Encoding
%% ----------------------------------------------------
\subsection{Delta Encoding}
Record the difference between values that follow each other in the same column.
The base value can be stored in-line or in a separate look-up table.
Can be combined with RLE to get even better compression ratios.

%% ----------------------------------------------------
%% Incremental Encoding
%% ----------------------------------------------------
\subsection{Incremental Encoding}
Type of delta encoding whereby common prefixes or suffixes and their lengths are recorded 
so that they need not be duplicated. This works best with sorted data.

%% ----------------------------------------------------
%% Mostly Encoding
%% ----------------------------------------------------
\subsection{Mostly Encoding}
When the values for an attribute are ``mostly'' less than the largest size, you can store 
them as a smaller data type.
The remaining values that cannot be compressed are stored in their raw form in a separate 
space.
From \dbSys{Amazon Redshift}.

%% ==================================================================
%% DICTIONARY COMPRESSION
%% ==================================================================
\section{Dictionary Compression}
The most common database compression scheme is dictionary encoding~\cite{p283-binnig}.
The DBMS replaces frequent pattern with smaller codes.
A dictionary compression scheme needs to support fast encoding/decoding, as well as range queries.

%% ----------------------------------------------------
%% Dictionary Construction
%% ----------------------------------------------------
\subsection{Dictionary Construction}

\textbf{Approach \#1 -- All-at-Once:}
\begin{itemize}
    \item
    Compute the dictionary for all the tuples at a given point of time.
    
    \item
    New Tuples must use a separate dictionary or all the tuples must be recomputed.
\end{itemize}

\textbf{Approach \#2 -- Incremental:}
\begin{itemize}
    \item
    Merge new tuples in with an existing dictionary.
    
    \item
    Likely requires re-encoding to existing tuples.
\end{itemize}

%% ----------------------------------------------------
%% Dictionary Scope
%% ----------------------------------------------------
\subsection{Dictionary Scope}

\textbf{Approach \#1 -- Lock Level:}
\begin{itemize}
    \item
    Only include a subset of tuples within a single table.
    
    \item
    Build dictionaries per block.
    
    \item
    Potentially lower compression ratio, but adding new tuples is easier.
\end{itemize}

\textbf{Approach \#2 -- Table Level:}
\begin{itemize}
    \item
    Construct a dictionary for the entire table.
    
    \item
    Better compression ratio, but expensive to update.
\end{itemize}

\textbf{Approach \#3 -- Multi-Table:}
\begin{itemize}
    \item
    Can be either subset or entire tables.
    
    \item
    Sometimes helps with joins and set operations.
    For example, if two tables are connected via a foreign key and if the columns in both 
    tables use the same dictionary, them the DBMS does not have to re-encode the values when it 
    joins those tables together.
\end{itemize}

%% ----------------------------------------------------
%% Dictionary Scope
%% ----------------------------------------------------
%     \subsection{Attribute Encoding}
%     \begin{itemize}
%         \item Instead of storing a single value per dictionary entry, store entries that span 
% attributes
%     \end{itemize}

%% ----------------------------------------------------
%% Encoding and Decoding
%% ----------------------------------------------------
\subsection{Encoding and Decoding}
A dictionary needs to support two operations:
\begin{itemize}
    \item \textbf{Encoding}:
    For a given uncompressed value, convert it into its compressed form.
    
    \item \textbf{Decode}:
    For a given compressed value, convert it back into its original form.
    
    \item
    No magic hash function will do this for us.
    
    \item
    Requires two data structures to support operations in both directions.
    The encoded values need to support sorting in the same order as original values.
\end{itemize}

%% ----------------------------------------------------
%% Dictionary Implementations
%% ----------------------------------------------------
\subsection{Dictionary Implementations}

\textbf{Approach \#1 -- Hash Table:}
\begin{itemize}
    \item
    Fast and compact.
    
    \item
    Unable to support range and prefix queries.
\end{itemize}

\textbf{Approach \#1 -- B+ Tree:}
\begin{itemize}
    \item
    Slower than a hash table and takes more memory.
    \item
    Can support range and prefix queries.
    
    \item
    Use a shared leaf node design to minimize the amount of redundant data~\cite{p283-binnig}.
\end{itemize}


% ==================================================================
% BIBLIOGRAPHY
% ==================================================================
\newpage
\bibliographystyle{abbrvnat}
\bibliography{11-compression}

\end{document}
