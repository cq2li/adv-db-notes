\documentclass[11pt]{article}

\newcommand{\lecturenumber}{13}
\newcommand{\lecturename}{Networking Protocols}
\newcommand{\lecturedata}{2018-03-05}

\usepackage{../dbnotes}

\begin{document}

\maketitle
\thispagestyle{plain}

%% ==================================================================
%% DATABASE ACCESS
%% ==================================================================
\section{Database Access}
In class, all the demos are featured through a terminal client. SQL queries are written by hand and 
results are printed to the terminal. In practice, real programs access the database through an API. 
There are three broad categories of database access APIs: \textit{Direct Access (DBMS-specific)}, 
\textit{Open Database Connectivity (ODBC)}, and \textit{Java Database Connectivity (JDBC)}.

%% ----------------------------------------------------
%% Open Database Connectivity
%% ----------------------------------------------------
\subsection*{Open Database Connectivity (ODBC)}
ODBC was originally developed in the early 1990s by Microsoft and Simba Technology. It is now the 
standard API for accessing a DBMS and is designed to be independent of the DBMS and OS. ODBC is 
mostly used for C/C++ development. Every major relational DBMS now has an ODBC implementation.

ODBC is based on the device driver model. The \textit{driver} encapsulates the logic needed to 
convert a standard set of commands into the DBMS-specific calls. The ODBC driver then uses a wire 
protocol (DBMS-specific) to communicate with the database.

ODBC can also emulate features that are required by the standard API, but not provided by the DBMS 
itself. For instance, the DBMS may not support cursors. However, ODBC provides a cursor API that 
requires implementation on the driver's side. Users can then use ODBC's cursor API as if the DBMS 
supports cursors.

%% ----------------------------------------------------
%% Java Database Connectivity
%% ----------------------------------------------------
\subsection*{Java Database Connectivity (JDBC)}
JDBC can be considered a version of ODBC for the programming language Java instead of C/C++. It is 
developed by Sun Microsystems in 1997 to provide a standard API for connecting a Java program with a 
DBMS. Similar to ODBC, it exposes a standard set of APIs and use drivers to communicate with the 
DBMS. 

There are four different ways to implement JDBC:
\begin{itemize}
    \item \textbf{JDBC-ODBC Bridge}:
    Convert JDBC method calls into ODBC function calls. This approach provides no actual means of 
    communicating with the DBMS using Java.
    
    \item \textbf{Native-API Driver}:
    Convert JDBC method calls into native calls of the target 
    DBMS API.
    
    \item \textbf{Network-Protocol Driver}:
    Driver connects to a middleware that converts JDBC calls into a vendor-specific DBMS protocol. 
    The middleware then communicates with the DBMS. The driver is not connected directly to the 
    DBMS.
    
    \item \textbf{Database Protocol Driver}:
    Pure Java implementation that converts JDBC calls directly into a vendor-specific DBMS protocol. 
    This is the most common implementation.
\end{itemize}

%% ==================================================================
%% DATABASE NETWORKING PROTOCOLS
%% ==================================================================
\section{Database Networking Protocols}
All major DBMSs implement their own proprietary wire protocol over TCP/IP. ODBC/JDBC masks the 
complexity of each individuals DBMSs wire protocol.

%% ----------------------------------------------------
%% Client/Server Interaction
%% ----------------------------------------------------
\subsection*{Client/Server Interaction}
The client connects to DBMS and begins authentication process, which may involve an SSL handshake. 
Then, the client sends a query to the DBMS. The DBMS executes the query, then serializes the results 
and sends it back to the client. The serialization is usually considered the bottleneck causing 
performance issues.

%% ----------------------------------------------------
%% Existing Protocols
%% ----------------------------------------------------
\subsection*{Existing Protocols}
Most newer systems implement one of the open-source DBMS wire protocols. The two most common are 
\dbSys{Postgres} and \dbSys{MySQL}. This allows a new DBMS to reuse the client existing drivers for 
these other systems without having to develop and support them. However, just because one DBMS 
``speaks'' another DBMS's wire protocol does not mean that it is compatible. The packets transferred 
through the wire are still DBMS-specific. The DBMSs need to support catalogs, SQL dialect, and 
other functionality that can process the packets correctly.

%% ==================================================================
%% PROTOCOL DESIGN SPACE
%% ==================================================================
\section{Protocol Design Space}
Protocol designs involve API layouts and data compression/serialization 
schemes~\cite{Raasveldt2017}.

%% ----------------------------------------------------
%% Row vs. Column Layout
%% ----------------------------------------------------
\subsection*{Row vs. Column Layout}
ODBC and JDBC are inherently row-oriented APIs. The DBMS packages tuples into messages one tuple 
at a time. Therefore, the client has to deserialize data one tuple at a time. 

However, modern data analysis software (e.g., Tensorflow, \dbSys{Spark}) operates on matrices and 
columns. Serializing/Deserializing one tuple at a time makes it inefficient for OLAP queries. To 
improve performance, one potential solution is to send data in vectors. The DBMS should send back 
batch of rows in a column-oriented fashion.

%% ----------------------------------------------------
%% Compression
%% ----------------------------------------------------
\subsection*{Compression}
If the database server has adopted a compression scheme, the driver client should use a 
corresponding compression scheme as well to communicate with the server. The DBMS and client can use 
either \textit{Naive Compression} or \textit{Columnar-Specific Encoding}.

From a software engineering perspective, one can argue that Naive Compression performs better 
than Columnar-Specific Encoding for network communication. This is because naive 
compression is agnostic to the shape of data. For instance, the server can just serialize the data, 
compress it using gzip, and send it to the client; the client then retrieves the data using gzip as 
well. This requires no implementation on the client side as it uses an available library. If the 
server uses Columnar-Specific Encoding, it has to figure out what the best compression scheme on the 
fly. Additionally, the client has to implement a corresponding scheme as well. 

When the network is slow, the DBMS should favor more heavyweight compression schemes to reduce the 
amount of data transferred on the wire. This is an obvious trade-off between CPU usage and 
bandwidth.

%% ----------------------------------------------------
%% Data Serialization
%% ----------------------------------------------------
\subsection*{Data Serialization}
Data are serialized into packets to be transferred on the wire. There are two ways to serialize the data:

\begin{itemize}
    \item \textbf{Binary Encoding}:
    Represent data in its binary form. The DBMS can implement its own 
    binary encoding format or rely on existing libraries (e.g., ProtoBuf, Thrift). The closer the 
    serialized format is to the DBMS's binary format, the lower the overhead to serialize. Since 
    endianness can differ between the DBMS and the client, the client has to handle endian 
    conversion as well. 
    
    \item \textbf{Text Encoding}:
    Convert all binary values into strings. Since everything is a 
    string, the client does not have to worry about endianness. The downside of the method is more 
    data storage.
\end{itemize}

%% ----------------------------------------------------
%% String Handling
%% ----------------------------------------------------
\subsection*{String Handling}
There are three approaches to handling strings for the DBMS and driver:

\begin{itemize}
    \item \textbf{Null Termination}:
    Store a null byte ('\textbackslash 0') to denote the end of a 
    string. Client has to scan entire string to find the end of the string.
    
    \item \textbf{Length Prefixes}:
    Add the length of the string at the beginning of the bytes.
    
    \item \textbf{Fixed Width}:
    Pad every string to be the max size of the attribute.
\end{itemize}

There is no particular string handling approach that stands out above the others; all three 
approaches perform differently in different settings.

%% ==================================================================
%% KERNEL BYPASS METHODS
%% ==================================================================
\section{Kernel Bypass Methods}
The DBMS's network protocol implementation is not the only source of slowdown; a lot of network 
operations rely on the OS. 

The OS's TCP/IP stack is slow due to a number of reasons. To transfer messages, the OS has to 
perform expensive context switches and interrupts. Moreover, in order to send a message, the OS 
copies the packet to prevent it being deallocated before the packet reaches the networking hardware 
layer. The OS also maintains its own latches for its own data structures, which can become potential 
bottlenecks as well for the DBMS/driver.

Therefore, some DBMS try to avoid the OS. We use kernel bypass methods to allows the 
system to get data directly from the Network Interface Controller (NIC) into the DBMS address space. 
This saves the need for data copying or OS TCP/IP stack since the DBMS will be talking directly to 
the hardware layer. There are two kernel bypass methods:

%% ----------------------------------------------------
%% Data Plan Development Kit
%% ----------------------------------------------------
\subsection*{Data Plane Development Kit (DPDK)}
DPDK is a set of libraries that allows programs to access NIC directly. It treats the NIC as a 
bare metal device. This approach requires the DBMS code to do more to manage memory and buffers. An 
example of a DBMS that uses this is \dbSys{ScyllaDB}.

%% ----------------------------------------------------
%% Remote Direct Memory Access
%% ----------------------------------------------------
\subsection*{Remote Direct Memory Access (RDMA)}
RDMA libraries allow reading and writing memory directly on a remote host without going through 
the OS. The client needs to know the correct address of the data that it wants to access; the server 
is unaware that memory is being accessed remotely. \dbSys{Oracle RAC} and \dbSys{Microsoft FaRM} are 
DBMSs that use RDMA.

% ==================================================================
% BIBLIOGRAPHY
% ==================================================================
\newpage
\bibliographystyle{abbrvnat}
\bibliography{13-networking}

\end{document}

