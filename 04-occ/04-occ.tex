\documentclass[11pt]{article}

\newcommand{\lecturenumber}{04}
\newcommand{\lecturename}{Optimistic Concurrency Control}
\newcommand{\lecturedata}{2018-01-29}

\usepackage{../dbnotes}

\begin{document}

\maketitle
\thispagestyle{plain}

%% ==================================================================
%% STORED PROCEDURES
%% ==================================================================

\section{Stored Procedures}
Disk stalls are (almost) gone when execuing transactions in an in-memory DBMS.
But there are still other stalls when an application uses \textbf{conversational} API to 
execute queries on DBMS (e.g., JDBC/ODBC, wire protocols).
Solutions to this problem:
\begin{enumerate}
    \item Prepared Statements
    \item Query Batches
    \item Stored Procedures
\end{enumerate}

\textbf{Prepared Statements:}
Provide the DBMS the SQL statement ahead of time and assign it to a name/handle.
Can invoke that query just by using that name.
This removes SQL parsing, binding, and planning (sometimes).
        
\textbf{Query Batches:}
Invoke multiple queries per network message.
This reduces the number of network roundtrips
        
\textbf{Stored Procedures:}
A group of queries that form a logical unit and perform a particular task on behalf of an 
application directly inside of the DBMS.
The application can then invoke the transaction as if it was an RPC.
This removes both preparation and network stalls.

Advantages of Stored Procedures:
\begin{itemize}
    \item
    Reduce the number of round trips between application and database servers.
    
    \item
    Increased performance because queries are pre-compiled and stored in DBMS.
    
    \item
    Procedure reuse across applications.
    
    \item
    Transpaerent sever-side transaction restarts on conflicts.
\end{itemize}

Disadvantages of Stored Procedures:
\begin{itemize}
    \item
    Not as many developers know how to write stored procedure code.
    
    \item
    Outside the scope of the application so it is difficult to manage versions and hard to 
    debug.
    
    \item
    Probably not to be portable to other DBMSs.
    
    \item
    DBAs usually do not give permissions out freely, so it makes it difficult for developers to 
    constantly update their stored procedures.
\end{itemize}

%% ==================================================================
%% Concurrency Control
%% ==================================================================
\section{Concurrency Control}
A DBMS' concurrency control protocol to allow transactions to access a database in a 
multi-programmed 
fashion while preserving the illusion that each of them is executing alone on a dedicated system.
The goal is to have the effect of a group of transactions on the database's state is equivalent to 
any 
serial execution of all transactions.
    
Concurrency Control Schemes
\begin{enumerate}
    \item \textbf{Two-Phase Locking (Pessimistic):}
    Assume transactions will conflict so they must acquire locks on database objects before they 
are 
    allowed to access them.
    
    \item \textbf{Timestamp Ordering (Optimistic):}
    Assume that conflicts are rare so transactions do not need to first acquire locks on database 
objects 
    and instead check for conflicts at commit time.
\end{enumerate}

% Two-Phase Locking
% \begin{itemize}
%     \item \textbf{Deadlock Detection:}
%     If deadlock is found, use a heuristic to decide what transaction to kill in order to break 
%     deadlock.
%     
%     \item \textbf{Deadlock Prevention:}
%     If lock is not available, then make a decision about how to proceed.
%     \end{itemize}
% \end{itemize}

%% ==================================================================
%% TIMESTAMP ORDERING AND OCC
%% ==================================================================
\section{Timestamp Ordering Concurrency Control}
Use timestamps to determine the order of transactions.

%% ----------------------------------------------------
%% Basic T/O Protocol
%% ----------------------------------------------------
\subsection*{Basic T/O Protocol}
\begin{itemize}
    \item
    Every transaction is assigned a unique timestamp when they arrive in the system.
    
    \item
    The DBMS maintains separate timestamps in each tuple's header of the last transaction that read 
    that tuple or wrote to it.
    
    \item
    Each transaction check for conflicts on each read/write by comparing their timestamp with the 
    timestamp of the tuple they are accessing.
    
    \item
    The DBMS needs copy a tuple into the transaction's private workspace when reading a tuple to 
    ensure repeatable reads.
\end{itemize}

%% ----------------------------------------------------
%% Optimistic Concurrency Control
%% ----------------------------------------------------
\subsection*{Optimistic Concurrency Control (OCC)}
Store all changes in private workspace.
Check for conflicts at commit time and then merge.
First proposed in 1981 at CMU by H. T. Kung ~\cite{p213-kung}.

\textbf{Three Phases:}
\begin{itemize}
    \item \textit{Read Phase:}
    Transaction's copy tuples accessed to private work space to ensure repeatable reads, and keep 
    track of read/write sets.
    
    \item \textit{Validation Phase:}
    When the transaction invokes \sql{COMMIT}, the DBMS checks if it conflicts with other 
transactions.
    Parallel validation means that each transaction must check the read/write set of other 
transactions 
    that are trying to validate at the same time. Each transaction has to acquire locks for its 
write set 
    records in some global order. Original OCC uses serial validation.
    \begin{itemize}
        \item Backward Validation:
        Check whether the committing transaction intersects its read/write sets with those of any 
transactions that         have \textbf{already} committed.
        
        \item Forward Validation:
        Check whether the commiting transaction intersects its read/write sets with any active 
transactions that         have \textbf{not} yet committed.
    \end{itemize}
    
    \item \textit{Write Phase:}
    The DBMS Propogates the changes in the transactions write set to the database and makes them 
    visible to other tnxs items. As each record is updated, the transaction releases the lock 
acquired     during the Validation Phase
\end{itemize}

\textbf{Timestamp Allocation:}~\cite{p209-yu}
\begin{itemize}
    \item \textit{Mutex:}
    Worst option. Mutexes are terrible.
    
    \item \textit{Atomic Addition:}
    Use compare-and-swap to increment a single global counter. Requires cache invalidation on write.
    
    \item \textit{Batched Atomic Addition:}
    Needs a back-off mechanism to prevent fast burn.

    \item \textit{Hardware Clock:}
    The CPU maintains an internal clock (not wall clock) that is synchronized across all coreds.
    Intel only. Not sure if it will exist in future CPUs.
    
    \item \textit{Hardware Counter:}
    Single global counter maintained in hardware. Not implemented in any existing CPUs.
\end{itemize}

%% ==================================================================
%% SILO OCC
%% ==================================================================
\section{Silo OCC}
The \dbSys{Silo} DBMS~\cite{tu-sosp2013,silo} is an influential in-memory DBMS developed by Harvard 
and MIT. It is a single-node OLTP system that uses Serializable OCC with parallel backward 
validation. All transactions execute as stored procedures.

The twp key ideas of Silo is that it seeks to avoid all writes to shared-memory for read-only 
transactions and that it uses batched timestamp allocation using epochs.

Implementation Details:
\begin{itemize}
    \item \textbf{Epochs}
    \begin{itemize}
        \item
        Time is sliced into fixed-length epochs (40~ms).
        
        \item
        All transactions that start within the same epoch will be committed together at the end of 
        the epoch.
        
        \item
        Transactions that span an epoch have to refresh themselves to be carried over into the next 
        epoch.
        
        \item
        Worker threads only need to synchronize at the beginning and end of each epoch.
    \end{itemize}
    
    \item \textbf{Transaction IDs}
    \begin{itemize}
        \item
        Each worker thread generates a unique transaction id based on the current epoch number and 
        the next value in its assigned batch.
    \end{itemize}
    
    \item \textbf{Garbage Collection}
    \begin{itemize}
        \item
        Cooperative threads GC.
        
        \item
        Each worker thread marks a deleted object with a \textbf{reclamation epoch}.
    \end{itemize}
    
    \item \textbf{Range Queries}
    \begin{itemize}
        \item
        The DBMSs handles phantoms by tracking the transactions scan set on indexes.
        
        \item
        Re-execute scans in the validation phase to see whether the index has changed.
        
        \item
        Have to include virtual entries for keys that do not exist in the index to prevent two 
        threads from trying to insert the same key.
    \end{itemize}
\end{itemize}

% ==================================================================
% BIBLIOGRAPHY
% ==================================================================
\newpage
\bibliographystyle{abbrvnat}
\bibliography{04-occ}

\end{document}
