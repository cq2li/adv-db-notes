\documentclass[11pt]{article}

\newcommand{\lecturenumber}{04}
\newcommand{\lecturename}{Optimistic Concurrency Control}
\newcommand{\lecturedata}{2018-01-29}

\usepackage{../dbnotes}

\begin{document}

\maketitle
\thispagestyle{plain}

%% ==================================================================
%% STORED PROCEDURES
%% ==================================================================

\section{Stored Procedures}
Disk stalls are (almost) gone when execuing txns in an in-memory DBMS.
But there are still other stalls when an application uses \textbf{conversational} API to 
execute queries on DBMS (e.g., JDBC/ODBC, wire protocols).
Solutions to this problem:
\begin{enumerate}
    \item Prepared Statements
    \item Query Batches
    \item Stored Procedures
\end{enumerate}

\textbf{Prepared Statements:}
Provide the DBMS the SQL statement ahead of time and assign it to a name/handle.
Can invoke that query just by using that name.
This removes SQL parsing, binding, and planning (sometimes).
        
\textbf{Query Batches:}
Invoke multiple queries per network message.
This reduces the number of network roundtrips
        
\textbf{Stored Procedures:}
A group of queries that form a logical unit and perform a particular task on behalf of an 
application directly inside of the DBMS.
The application can then invoke the transaction as if it was an RPC.
This removes both preparation and network stalls.

Advantages of Stored Procedures:
\begin{itemize}
    \item
    Reduce the number of round trips between application and database servers.
    
    \item
    Increased performance because queries are pre-compiled and stored in DBMS.
    
    \item
    Procedure reuse across applications.
    
    \item
    Transpaerent sever-side txn restarts on conflicts.
\end{itemize}

Disadvantages of Stored Procedures:
\begin{itemize}
    \item
    Not as many developers know how to write stored procedure code.
    
    \item
    Outside the scope of the application so it is difficult to manage versions and hard to 
    debug.
    
    \item
    Probably not to be portable to other DBMSs.
    
    \item
    DBAs usually do not give permissions out freely, so it makes it difficult for developers to 
    constantly update their stored procedures.
\end{itemize}

%% ==================================================================
%% Concurrency Control
%% ==================================================================
\section{Concurrency Control}
A DBMS' concurrency control protocol to allow txns to access a database in a multi-programmed 
fashion while preserving the illusion that each of them is executing alone on a dedicated system.
The goal is to have the effect of a group of txns on the database's state is equivalent to any 
serial execution of all txns.
    
Concurrency Control Schemes
\begin{enumerate}
    \item \textbf{Two-Phase Locking (Pessimistic):}
    Assume txns will conflict so they must acquire locks on database objects before they are 
    allowed to access them.
    
    \item \textbf{Timestamp Ordering (Optimistic):}
    Assume that conflicts are rare so txns do not need to first acquire locks on database objects 
    and instead check for conflicts at commit time.
\end{enumerate}

% Two-Phase Locking
% \begin{itemize}
%     \item \textbf{Deadlock Detection:}
%     If deadlock is found, use a heuristic to decide what txn to kill in order to break 
%     deadlock.
%     
%     \item \textbf{Deadlock Prevention:}
%     If lock is not available, then make a decision about how to proceed.
%     \end{itemize}
% \end{itemize}

%% ==================================================================
%% TIMESTAMP ORDERING AND OCC
%% ==================================================================
\section{Timestamp Ordering Concurrency Control}
\textbf{Basic T/O Protocol:}
\begin{itemize}
    \item
    Every transaction is assigned a unique timestamp when they arrive in the system.
    
    \item
    The DBMS maintains separate timestamps in each tuple's header of the last transaction that read 
    that tuple or wrote to it.
    
    \item
    Each transaction check for conflicts on each read/write by comparing their timestamp with the 
    timestamp of the tuple they are accessing.
    
    \item
    The DBMS needs copy a tuple into the transaction's private workspace when reading a tuple to 
    ensure repeatable reads.
\end{itemize}
%     \item Optimistic Concurrency Control (OCC)
%     \begin{itemize}
%         \item Store all changes in private workspace
%         \item Check for conflicts at commit time and then merge
%         \item First proposed in 1981 at CMU by H. T. Kung ~\cite{p213-kung}
%         \item Read Phase: Transaction's copy tuples accessed to private work space to ensure repeatable reads, and keep track of read/write sets
%         \item Validation Phase: When the txn invokes \sql{COMMIT}, the DBMS checks if it conflicts with other transactions
%         \begin{itemize}
%             \item Backward Validation: Check whether the committing txn intersects its read/write sets with those of any txns that have \textbf{already} committed
%             \item Forward Validation: Check whether the commiting txn intersects its read/write sets with any active txns that have \textbf{not} yet committed
%             \item Original OCC uses serial Validation
%             \item Parallel validation means that each txn must check the read/write set of other txns that are trying to validate at the same time. Each txn has to acquire locks for its write set records in some global order
% 
%         \end{itemize}
%         \item Write Phase
%         \begin{itemize}
%             \item The DBMS Propogates the changes in the txns write set to the database and makes them visible to other tnxs
%             items
%             \item As each record is updated, the txn releases the lock acquired during the Validation Phase
%         \end{itemize}
%         \item Timestamp Allocation~\cite{p209-yu}
%         \begin{itemize}
%             \item Mutex: Worst option. Mutexes are the "Hitler of Concurrency"
%             \item Atomic Addition: Requires cache invalidation on write
%             \item Batched atomic addition: Needs a back-off mechanism to prevent fast burn
%             \item Hardware Clock: Not sure if it will exist in future CPU's
%             \item Hardware Counter: Not implemented in existing CPU's
%         \end{itemize}
% \end{itemize}

%% ==================================================================
%% SILO OCC
%% ==================================================================
\section{Silo OCC}
\begin{itemize}
    \item \dbSys{Silo} DBMS ~\cite{tu-sosp2013}
%     \begin{itemize}
%         \item Single-node in-memory OLTP DBMSs
%         \item Serializable OCC with parallel backward Validation
%         \item Stored procedure only API
%         \item No writes to shared-memory for read txns
%         \item Batched timestamp allocation using epochs
%     \end{itemize}
%     \item Epochs
%     \begin{itemize}
%         \item Time is sliced into fixed-length epochs (40ms)
%         \item All txns that start within the same epoch will be committed together at the end of the epoch
%         \item Txns that span an epoch have to refresh themselves to be carried over into the next epoch
%         \item Worker threads only need to synchronize at the beginning and end of each epoch
%     \end{itemize}
%     \item Transaction IDs
%     \begin{itemize}
%         \item Each worker thread generates a unique txn id based on the current epoch number and the next value in its assigned batch
%     \end{itemize}
%     \item Garbage Collection
%     \begin{itemize}
%         \item Cooperative threads GC
%         \item Each worker thread marks a deleted object with a \textbf{reclamation epoch}
%     \end{itemize}
%     \item Range queries
%     \begin{itemize}
%         \item DBMSs handles phantoms by tracking the txns scan set on indexes
%         \item Re-execute scans in the validation phase to see whether the index has changed
%         \item Have to include virtual entries for keys that do not exist in the index
%     \end{itemize}
\end{itemize}

% ==================================================================
% BIBLIOGRAPHY
% ==================================================================
\newpage
\bibliographystyle{abbrvnat}
\bibliography{04-occ}

\end{document}
