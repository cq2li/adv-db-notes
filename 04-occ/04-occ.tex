\documentclass[11pt]{article}

\newcommand{\lecturenumber}{4}
\newcommand{\lecturename}{Optimistic Concurrency Control}
\newcommand{\lecturedata}{2018-01-29}

\usepackage{../dbnotes}

\begin{document}

\maketitle
\thispagestyle{plain}

\section{Stored Procedures}
\begin{itemize}
    \item Disk stalls are (almost) gone when execuing txns in an in-memory DBMS
    \item There are still other stalls when an app uses \textbf{conversational} API to execute queries on DBMS
    \item Solutions
    \begin{itemize}
        \item Prepared Statements: Removes query preparaatiion overhead
        \item Query Batches: Reduces the number of network roundtrips
        \item Stored Procedures: Removes both preparation and network stalls
    \end{itemize}
    \item \textbf{Stored Procedures}: A group of queries that form a logical unit and perform a particular task on behalf of an application directly inside of the DBMS
    \item Advantages
    \begin{itemize}
        \item Reduce the number of round trips between application and database servers
        \item Increased performance because queries are pre-compiled and stored in DBMS
        \item Procedure reuse across applications
        \item Sever-side txn restarts on conflicts
    \end{itemize}
    \item Disadvantages
    \begin{itemize}
        \item Not as many developers know how to write SQL/PSM code
        \item Outside the scope of the application so it is difficult to manage versions and hard to debug
        \item Probably not to be portable to other DBMSs
        \item DBAs usually don't give permissions out freely
    \end{itemize}
\end{itemize}

\section{Concurrency Control}
\begin{itemize}
    \item The protocol to allow txns to access a database in a multi-programmed fashion while preserving the illusion that each of them is executing alone on a dedicated system
    \item The goal is to have the effect of a group of txns on the database's state is equivalent to any serial execution of all txns
    \item Concurrency Control Schemes
    \begin{enumerate}
        \item Two-Phase Locking (Pessimistic) ...
        \item Timestamp Ordering (Optimistic) ...
    \end{enumerate}
    \item Two-Phase Locking
    \begin{itemize}
        \item Deadlock Detection: If deadlock is found, use a heuristic to decide what txn to kill in order to break deadlock
        \item Deadlock Prevention: If lock is not available, then make a decision about how to proceed
    \end{itemize}
\end{itemize}

\section{Timestamp Ordering and OCC}
\begin{itemize}
    \item Basic T/O
    \begin{itemize}
        \item Check for conflicts on each read/write
        \item Copy tuples on each access to ensure repeatable reads
    \end{itemize}
    \item Optimistic Concurrency Control (OCC)
    \begin{itemize}
        \item Store all changes in private workspace
        \item Check for conflicts at commit time and then merge
        \item First proposed in 1981 at CMU by H. T. Kung ~\cite{p213-kung}
        \item Read Phase: ...
        \item Validation Phase: ...
        \begin{itemize}
            \item Backward Validation: Check whether the committing txn intersects its read/write sets with those of any txns that have \textbf{already} committed
            \item Forward Validation: Check whether the commiting txn intersects its read/write sets with any active txns that have \textbf{not} yet committed
            \item Original OCC uses serial Validation
            \item Parallel validation means that each txn must check the read/write set of other txns that are trying to validate at the same time. Each txn has to acquire locks for its write set records in some global order

        \end{itemize}
        \item Write Phase
        \begin{itemize}
            \item The DBMS Propogates the changes in the txns write set to the database and makes them visible to other tnxs
            items
            \item ...
        \end{itemize}
        \item Timestamp Allocation~\cite{p209-yu}
        \begin{itemize}
            \item Mutex: Worst option. Mutexes are the "Hitler of Concurrency"
            \item Atomic Addition: Requires cache invalidation on write
            \item Batched atomic addition: Needs a back-off mechanism to prevent fast burn
            \item Hardware Clock: Not sure if it will exist in future CPU's
            \item Hardware Counter: Not implemented in existing CPU's
        \end{itemize}
    \end{itemize}
\end{itemize}

\section{Silo OCC}
\begin{itemize}
    \item \dbSys{Silo} DBMS ~\cite{tu-sosp2013}
    \begin{itemize}
        \item Single-node in-memory OLTP DBMSs
        \item Serializable OCC with parallel backward Validation
        \item Stored procedure only API
        \item No writes to shared-memory for read txns
        \item Batched timestamp allocation using epochs
    \end{itemize}
    \item Epochs
    \begin{itemize}
        \item Time is sliced into fixed-length epochs (40ms)
        \item All txns that start within the same epoch will be committed together at the end of the epoch
        \item Txns that span an epoch have to refresh themselves to be carried over into the next epoch
        \item Worker threads only need to synchronize at the beginning and end of each epoch
    \end{itemize}
    \item Transaction IDs
    \begin{itemize}
        \item Each worker thread generates a unique txn id based on the current epoch number and the next value in its assigned batch
    \end{itemize}
    \item Garbage Collection
    \begin{itemize}
        \item Cooperative threads GC
        \item Each worker thread marks a deleted object with a \textbf{reclamation epoch}
    \end{itemize}
    \item Range queries
    \begin{itemize}
        \item DBMSs handles phantoms by tracking the txns scan set on indexes
        \item Re-execute scans in the validation phase to see whether the index has changed
        \item Have to include virtual entries for keys that do not exist in the index
    \end{itemize}
\end{itemize}

\section{Parting Thoughts}
\begin{itemize}
    \item Trade-off between aborting txns early or later
    \begin{itemize}
        \item Early: Avoid wasted work for txns that will eventually abort, but has checking overhead
        \item Later: No runtime overhead but lots of wasted work under high contention
    \end{itemize}
    \item Silo is a very influential system
\end{itemize}

% ==================================================================
% BIBLIOGRAPHY
% ==================================================================
\newpage
\bibliographystyle{abbrvnat}
\bibliography{04-occ}

\end{document}
